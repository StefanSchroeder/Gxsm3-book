%/* Gxsm - Gnome X Scanning Microscopy
% * universal STM/AFM/SARLS/SPALEED/... controlling and
% * data analysis software:  Documentation
% * 
% * Copyright (C) 1999,2000,2001 Percy Zahl
% *
% * Authors: Percy Zahl <zahl@users.sf.net>
% * additional features: Andreas Klust <klust@users.sf.net>
% * WWW Home: http://gxsm.sf.net
% *
% * This program is free software; you can redistribute it and/or modify
% * it under the terms of the GNU General Public License as published by
% * the Free Software Foundation; either version 2 of the License, or
% * (at your option) any later version.
% *
% * This program is distributed in the hope that it will be useful,
% * but WITHOUT ANY WARRANTY; without even the implied warranty of
% * MERCHANTABILITY or FITNESS FOR A PARTICULAR PURPOSE.  See the
% * GNU General Public License for more details.
% *
% * You should have received a copy of the GNU General Public License
% * along with this program; if not, write to the Free Software
% * Foundation, Inc., 59 Temple Place, Suite 330, Boston, MA 02111-1307, USA.
% */


\chapter{Configuration}
\label{ch:config}
\index{Preferences}
\index{Configuration|see{Preferences}}
\index{GConf|see{Preferences}}
\index{Druids|see{Preferences}}

All your settings are saved using the gconf database\footnote{GConf:
  try the gconf-editor to explore the GXSM settings, path:
  /apps/gxsm2!}.

The configuration druid uses the gconf-database and the save-values
command will rewrite it, just like the \filename{Accept} and
\filename{OK} button in the preferences.

With special care the gconf database can be edited, to set defaultvalues.
To restore a totally screwed configuration simply remove it. 
The next time you start \Gxsm\ the configuration wizard will pop
up to create your settings from the defaults.

With Settings/Preferences (dt: Einstellungen/Einstellungen) \Gxsm\ can
be configured during runtime, the changes are (as seen above) 
saved in \filename{\~/.gnome/gxsm}. This file can be copied to any new user,
who wants to share the same instrument.

Any of the settings provides a small help text for your information.


\section{Hardware folder}
\label{sec:conf:hardware}
\index{Preferences!Hardware}

\begin{figure}[hbt]
\center { \fighalf{PrefHardware}\fighalf{PrefHardwarePup} }
\caption{Screenshot of the \Gxsm\ preferences folder, hardware page.
Here the connected hardware type and device is specified. Shown are the settings for the Signal Ranger.}
\label{fig:config:hardware}
\end{figure}

\begin{description}
\item[Hardware/Card] Choose hardware: see also section \ref{sec:HwI-PlugIns} about Hardware-Interface (HwI) plug-ins for specific details.)
  \begin{description}
  \item[\filename{no}] no hardware connected, for data analysis, using internal dummy-mode.
  \item[\filename{SRanger:SPM}] Signal Ranger is connected, a running SPM DSP code is expected.
  \item[\filename{Innovative\_DSP:SPM}] A Intelligent Scanning Probe Hardware Module is expected at the device.
	  (pci32.o, pc31.o at /dev/pcdsp (see later) + running \filename{xafm.out} on DSP interface board
  \item[\filename{LAN\_RHK:SPM}] use the RHK internet device
  \item[\dots] some more experimental HwI modules may show up.
%  \filename{comedispm}: testing/development\\
%  \filename{spa}: intelligent SPA--LEED Hardware Module expected at device.
%  (pci32.o, pc31.o, dspbbspa.o,  dspspaemu.o /dev/pcdsp (see later) and running spa.out on 
%        DSP interface board in the case the DSP version is running )\\
%  \filename{ccd}: For a very special CCD-Modul (ccd.o)\\
  \end{description}
\item[Hardware/Device] Path to the device used by the specific hardware. Typical are \filename{/dev/sranger0}, \filename{/dev/pcdsp}, \dots
\end{description}

\clearpage
\section{Instrument folder}
\label{sec:conf:instr}
\index{Preferences!Instrument}

\begin{figure}[hbt]
\center \fighalf{PrefInstrument}
\caption{Gxsm preferences folder, instrument page}
\label{fig:config:instrument}
\end{figure}

\begin{description}
\item[Instrument/Type] Select one of STM, AFM, SARLS, SPALEED, CCD.  This
determines the \gg{look} of the main window.
\item[Instrument/Name] Any name you want to associate with your instrument. Please limit to 30 characters.
\end{description}

\clearpage
\subsection{Inst-SPM folder}
\index{Preferences!Instrument-SPM}

\begin{figure}[hbt]
\center \fighalf{PrefInstSPM}
\caption{\Gxsm\ preferences folder, instrument SPM page}
\label{GxsmPrefInstrumentSPM}
\end{figure}

\begin{description}
\item[Instrument/X,Y,ZPiezoAV] SPM Piezo sensitivity in A/V.
\item[Instrument/BiasGain] Gain of the bias-volatge.
\item[Instrument/nAmpere2Volt] Tunnel amplifier sensitivity: \filename{factor*Volt=nA}
\item[Instrument/nNewton2Volt] for AFM Setpoint in nN, 1 for 1nN = 1V
\item[Instrument/dHertz2Volt] NC AFM Setpoint
\item[Instrument/ScanOrgCenter] Specific to hardware setup%: see \ref{SR:config}, \ref{PCI32:config}
\item[Analog/DigRangeIn] positive maximum value for X,Y,Z,\dots AD-conversion
  with respect to VoltMaxIn. The converters have to be bipolar
  (e.g. $\pm 10\;$V)%: see \ref{SR:config}, \ref{PCI32:config}
\item[Analog/VoltMaxIn] AD Voltage corresponding to DigRangeIn%: see \ref{SR:config}, \ref{PCI32:config}
 \item[Analog/DigRangeOut] positive maximum value for X,Y,Z,\dots DA-conversion
  with respect to VoltMaxOut. The converters have to be bipolar
  (e.g. $\pm 10\;$V)%: see \ref{SR:config}, \ref{PCI32:config}
\item[Analog/VoltMaxOut] DA Voltage corresponding to DigRangeOut%: see \ref{SR:config}, \ref{PCI32:config}
\item[Analog/V1-V9] Piezoamplifier Settings, typically:  1,2,5,10,15.
\item[Analog/VX/Y/Zdefault] Preferences at for programm startup gain selections
\item[Analog/VX0/Y0/Z0default] Preferences for programm startup gain selections for Offset, only essential if analog offset adding is used.
\end{description}

\clearpage
\subsection{Inst-SPA folder}
\index{Preferences!Instrument-SPA}
%\begin{figure}[hbt]
%\center \fighalf{PrefInstSPA}
%\caption{\Gxsm\ Preferences Folder, Instrument SPM Page}
%\label{GxsmPrefInstrumentSPA}
%\end{figure}

\begin{description}
\item[Instrument/X,YCalibV] SPALEED: X,Y calibration factor, 1V at DA
  $\rightarrow$ $\pm 10\;$V resp. $15\;$V at octopol front/back.
\item[Instrument/EnergyCalibVeV] factor*Volt = energy in eV
\item[Instrument/Sensitivity] $\textrm{BZ} = \textrm{U} \cdot \textrm{Sensitivity} /
\sqrt{\textrm{Energy}[\textrm{eV}]} $
\item[Instrument/ThetaChGunInt] Half angle between channeltron and electron gun (intern).
\item[Instrument/ThetaChGunExt] unused.
\item[Sample/LayerDist] Atom layer distance of the sample in \AA ngstroem, 
        used for calculation of phase (energy in \filename{S}).
\item[Sample/UnitLen] unused.
\end{description}


\clearpage
\section{DataAq folder}
\index{Preferences!DataAq}

\begin{figure}[hbt]
\center { \fighalf{PrefDataAq}\fighalf{PrefDataAq2} }
\caption{Gxsm preferences folder, DataAq page. The shown settings are
	for a AFM setup driven by the SIgnal Ranger.}
\label{fig:config:dataaq}
\end{figure}

In this section deals with the assignment of data sources to \Gxsm\ 
channels.  The particular configuration depends on your data
acquisition hardware and is unique for each instrument.  The following
explanations are based on the Signal Ranger DSP board.

\GxsmNote{For historic reasons the analog data source channels are
grouped in blocks of four (A,B) inputs, this was due to the old PCI32
DSP board \footnote{The old PCI32 DSP card is still supported by \Gxsm,
but some limits apply and it's less and less tested with newer \Gxsm\ 
versions.}  has four A/D-converters with four input-lines directly
connected. The even older PC31 has only two A/D-connectors, but four
input-lines are adressed to each of them via a 4x-multiplexer, which
results in eight usable analog input lines. Name convention in the
x-resources is A and B for the converters and 1 to 4 for the four
multiplexer input lines (PC31).  The PCI32 has four converters which
are called A1, A2, A3 and A4.}

For the Signal Ranger board the hardware channel assignment to the
\Gxsm data sources is defined as shown here:

\begin{table}[hbt]
\begin{tabular}{|l|l|l|}
\hline
Channel Descriptor& Name       & Notes\\\hline\hline
DataAq/PIDSrcA1   & Topo,*     & Z: generated by the Feedback\\
DataAq/PIDSrcA2\dots4 & ---    & not used\\\hline
DataAq/DataSrcA1  & AIC5       & AIC5: I, Force, dFrq, \dots what is used as feedback signal\\
DataAq/DataSrcA2  & AIC0       & AIC0 input (Fric)\\
DataAq/DataSrcA3  & AIC1       & AIC1 input (Damp., opt. FUZZY source)\\
DataAq/DataSrcA4  & AIC2       & AIC2 input\\\hline
DataAq/DataSrcB1  & AIC3       & AIC3 input\\
DataAq/DataSrcB2  & AIC4       & AIC4 input\\
DataAq/DataSrcB3  & AIC5       & AIC5 input\\
DataAq/DataSrcB4  & AIC6       & AIC6 input\\\hline
DataAq/DataSrcC1  & dIdV       & LockIn 1st of AIC5 (32bit)\\
DataAq/DataSrcD1  & ddIdV      & LockIn 2nd of AIC5 (32bit)\\\hline
DataAq/DataSrcE1  & I0         & test:FB-Integrator (LockIn-I0 avg) (32bit) \\\hline
DataAq/DataSrcF1  & Counter    & SR-CoolRunner Counter if equipped (experimental) (32bit)\\\hline
\end{tabular}
  \caption{Signal Ranger board the hardware channel to \Gxsm\ data sources assignment definitions. I0, Counter are experimental and may change any time}
  \label{tab:preferences:dataaq}
\end{table}


The field ``Name'' can be used to give the input a real name instead
of ``AIC0'' you can used ``PLL-dF'' or what ever you like. Do not use
a ``,'' except for ``,*'' to set a default, the text behind it will be ignored.
\GxsmHint{For you hackers, check out the SR code, push\_area\_scan\_data() in\\
\GxsmFile{SRanger/TiCC-project-files/FB\_spmcontrol/FB\_spm\_areascan.c}.}

\clearpage
\section{Probe folder}
\index{Preferences!Probe}

-- obsolete --

\section{Folder Adj.}
\index{Preferences!Adjustments}

-- obsolete -- see Entry-Popup menu!

Configuration of Value-Slider: 

Adjustments.Name/min,max for Area (please care for  min $<$ max !!),
Adjustments.Name/step,page for stepwidth at Cursor, Click.

\section{Paths folder}
\index{Preferences!Paths}

\begin{description}
\item[Path/Logfiles] 
\item[Path/Data] 
\item[Path/RemoteFifo] path to remote fifo (read only by \Gxsm)
\item[Path/RemoteFifoOut] path to control fifo \filename{Echo} (wo by \Gxsm)
\item[Path/Plugins] Additional Plugin searchpath
\end{description}


\section{User folder}
\index{Preferences!User}

\begin{description}
\item[User/SaveWindowGeometry] Always \filename{false}, for future use\dots
\item[User/Unit] XYZ-Einheit: Choice of AA, nm, um, mm, BZ, sec, V, 1
\item[User/HiLoDelta] Checking distance of array for calculation of Min/Max for
        Autodisplay, 1 = all points visited.
\item[User/FileType] \menuentry{nc} (\menuentry{dat} possible, but out of date.)
\item[User/NameConvention] \menuentry{digit}: Auto enumeration with 001, 002, \dots, 
  \menuentry{alpha}: Auto enumeration with aaa, aab, \dots
\item[User/SliderControlType] \menuentry{slider}: Omicron Slider Control, \menuentry{mover}: Besocke
\item[User/Palette] path to palette image (may be altered at runtime) See also ...
%\item[User/] 
\end{description}

\section{GUI folder}
\index{Preferences!GUI}

\begin{description}
\item[GUI/layerfields] Select here, if you want use layered (3d) scans.
%\item[GUI/] 
\end{description}
