%/* Gxsm - Gnome X Scanning Microscopy
% * universal STM/AFM/SARLS/SPALEED/... controlling and
% * data analysis software:  Documentation
% * 
% * Copyright (C) 1999,2000,2001 Percy Zahl
% *
% * Authors: Percy Zahl <zahl@users.sf.net>
% * additional features: Andreas Klust <klust@users.sf.net>
% * WWW Home: http://gxsm.sf.net
% *
% * This program is free software; you can redistribute it and/or modify
% * it under the terms of the GNU General Public License as published by
% * the Free Software Foundation; either version 2 of the License, or
% * (at your option) any later version.
% *
% * This program is distributed in the hope that it will be useful,
% * but WITHOUT ANY WARRANTY; without even the implied warranty of
% * MERCHANTABILITY or FITNESS FOR A PARTICULAR PURPOSE.  See the
% * GNU General Public License for more details.
% *
% * You should have received a copy of the GNU General Public License
% * along with this program; if not, write to the Free Software
% * Foundation, Inc., 59 Temple Place, Suite 330, Boston, MA 02111-1307, USA.
% */


\chapter{Program Start}
\label{ch:start}

This chapter describes the very basics of \Gxsm\ usage: the start of the program.
In a modern Gnome3 (gnome-shell based) desktop environment with a proper installed gxsm3 you will find
the \Gxsm\ in the Aplication menu, can seek via gnome-sell command, etc. and place
it into your quick start application bar.

However, some of its behavior can be modified using the command line parameters described in section
\ref{sec:start:command-line-pars}.

\section{Command line parameters}
\label{sec:start:command-line-pars}
\index{Command line parameters}
\index{Starting GXSM-3}
To start from a command line just type -- no special options at all:\\
\GxsmCmd{gxsm3}\\
Or for temporary hardware disabling, very handy for data analysis on your system configured for hardware usage by default:\\
\GxsmCmd{gxsm3 -h no}

\GxsmHint{It is recommended to use this as default handler of \GxsmFile{*.nc} Mime-File types, i.e. used by Nautilus to launch \Gxsm\ while clicking a data file.} To list all understood command lien options type:

\GxsmCmd{gxsm3 -- --help}\\

{\small\begin{verbatim}
~$ gxsm3 --help
Usage:
gxsm3 [OPTION...] List of loadable file(s) .nc, ...

Help Options:
-?, --help                                 Show help options
--help-all                                 Show all help options
--help-gtk                                 Show GTK+ Options

Application Options:
-V, --version                              Show the application's version
-h, --hardware-card                        Hardware Card: no | ... (depends on available HwI plugins)
-d, --Hardware-DSPDev                      Hardware DSP Device Path: /dev/sranger0 | ... (depends on module type and index if multiple DSPs)
-u, --User-Unit                            XYZ Unit: AA | nm | um | mm | BZ | sec | V | 1 
-L, --logging-level                        Set Gxsm logging/monitor level. omit all loggings: 0, minimal logging: 1, default logging: 2, verbose logging: 3, ...
-m, --load-files-as-movie                  load file from command in one channel as movie
--disable-plugins                          Disable default plugin loading on startup
--force-configure                          Force to reconfigure Gxsm on startup
--force-rebuild-configuration-defaults     Forces to restore all GXSM values to build in defaults at startup
--write-gxsm-preferences-gschema           Generate Gxsm preferences gschema file on startup with build in defaults and exit
--write-gxsm-gl-preferences-gschema        Generate Gxsm GL preferences gschema file on startup with build in defaults and exit
--write-gxsm-pcs-gschema                   Generate Gxsm pcs gschema file on startup with build in defaults while execution
--write-gxsm-pcs-adj-gschema               Generate Gxsm pcs adjustements gschema file on startup with build in defaults while execution
-D, --debug-level=DN                       Set Gxsm debug level. 0: no debug output on console, 1: normal, 2: more verbose, ...5 increased verbosity
-P, --pi-debug-level=PDN                   Set Gxsm Plug-In debug level. 0: no debug output on console, 1: normal, 2: more verbose, ...5 increased verbosity
-s, --new-instance                         Start a new instance of gxsm3 -- not yet functional, use different user account via ssh -X... for now.
--display=DISPLAY                          X display to use

\end{verbatim}}

\begin{table}[hb]
  \begin{center}
    \begin{tabular}{|l|l|} \hline
      Parameter        & Description\\ \hline \hline    
        -h, --Hardware-Card={\textit{type}} & set up type of hardware\\
\multicolumn{1}{|r|}{ no } &  do not use any hardware, analysis/simulation mode\\
\multicolumn{1}{|r|}{ \dots } &  depends on available HwIs \\
\hline

        -u, --User-Unit={\textit{unit}} & XYZ unit\\
\multicolumn{1}{|r|}{ AA  } & \AA, $1e-10$~m \\
\multicolumn{1}{|r|}{ nm  } & nm, $1e-9$~m \\ \hline

       --disable-plugins & Disables the loading of plugins on startup\\
                         & mainly for debugging.\\ \hline

    \end{tabular}
    \caption{commando line parameters for calling Gxsm}
    \label{tab:command-line}
  \end{center}
\end{table}
\footnotetext[1]{This is the default, the device location can be configured using the \GxsmPref{Hardware}{Hardware/Device}}
\footnotetext[2]{The device location can be configured using the \GxsmPref{Hardware}{Hardware/Device}, use '/dev/sranger0'.}

Right after the options shown above you can list files for opening to the free channels.

All known formats (\ref{pi:WSxMio}\ref{pi:nanoimport}\ref{pi:OmicronIO}\ref{pi:primitiveimexport}\ref{pi:sdfimport}\ref{pi:UK2kimport}\ref{pi:gmeyerimexport}\ref{pi:spa4imexport}\ref{pi:GdatImExport}\ref{pi:PsiHDFimexport}\ref{pi:pngImExport}) are autodetected.


\chapter{The main window}
\label{ch:main}
\index{Main Window}

After startup, the main window appears.  The actual user interface provided by the main window depends on the configuration (compare Fig.~\ref{fig:screenshot:gxsm3-main}. \Gxsm\ can be configured for use with SPM techniques, which is the default.

%, or SPA-LEED.

\GxsmScreenShot{gxsm3-main}{Gxsm3 Main Scan Control Window.}

%\begin{figure}[hbt]
%\hfill\fighalf{GxsmSPM}\hfill\fighalf{GxsmSPALEED}\hfill
%\caption{Gxsm Main Window, \filename{SPM}-variant left, \filename{SPALEED}-variant right.}
%\label{fig:GxsmMainWindow}
%\end{figure}

The main window provides two different functions:  Firstly, it has a menu bar with pull-down menus.  These menus provide the user with the usual \menuentry{File} and \menuentry{Help} menus which can be found in practically every mouse-driven software piece.  Some of these pull-down menus interact with (\menuentry{Math}) or start-up (\menuentry{Windows}) other windows.  Secondly, the main menu contains a large number of control fields which can be used, e.g., to control an instrument, or just display certain parameters. These control fields are described in the following two sections.


\section{Understanding the main window's entries: SPM mode}
\label{sec:main:spmentries}

This section explains the contents of the main window (see Fig.~\ref{fig:screenshot:gxsm3-main}) for Gxsm running in SPM mode.  The main window contains from top to bottom the menubar, a taskbar, the scan parameter, view mode, file, and info/comment sections, and a status and progress bar. The scan parameter and info sections of the main window are used both for entering parameters during data taking and displaying them after loading data.

\subsection{Scan parameters}

Each scan or image is characterized by its size and resolution. The size, or \GxsmEmph{Range XY}, gives the scale of the image like the scale of a city map and denotes the height and width of the scanned area. The resolution is determined by either the distance between the single scan points/pixels given by \GxsmEmph{Steps XY} or the number of points in X and Y direction given by \GxsmEmph{Points XY}. Given two of these parameters, the third one can be computed. The check box \GxsmEmph{Calculate} determines, which of
them is calculated by \Gxsm. For instance, if \GxsmEmph{Steps} is checked, a change of \GxsmEmph{Range XY} results automagically in a new value for \GxsmEmph{Steps XY}.

The parameter \GxsmEmph{Offset XY} determines the distance of the zeropoint of the image coordinates from the center of the physical scanrange. The actual location of the zeropoint within the scan depends on the source of the data. If the data was acquired using \Gxsm, the zeropoint is the middle of the topmost line. Using \GxsmEmph{Rotation}, the imaged area can be rotated. Both inputs
using numeric values and the scrollbar are possible.

\Gxsm\ can be used to do spatially resolved spectroscopy (``probing'') and time dependent measurements (``movies''). Channels containing probing (or time dependent) data are essentially three-dimensional (3D) datasets. In these 3D datasets the X and Y coordinates correspond to the 2D position like in conventional SPM images. The third dimension can be the voltage V or the time t. \Gxsm\ displays only one slice corresponding to one V or one t value at a time. \GxsmEmph{Layers} denotes the number of points in the V direction, \GxsmEmph{Time} in the temporal direction.

\GxsmEmph{VRange Z} and \GxsmEmph{VOffset Z} are used for the visualization of the scan data. They do not influence the data itself. See also Sec.~\ref{bright-contrast}.

\subsection{File and user information}

For the users convenience, the filenames for saving new data are automatically generated. The filename is set together using the
\GxsmEmph{Basename} and the scan number. The default for the \GxsmEmph{Basename} is the login name of the \Gxsm\ user. The scheme used
for generating the filename from the scan number can be configured in the Preferences on the tab \GxsmEmph{User}. The scan number is followed by ``-M-'' if the image contains additional information besides the bare 2D image like events and point probes. The next part of the file name indicates the scan direction: Xp or Xm. Finally, the channel name is attached to the file name. The channel name can be configured on the tab \GxsmEmph{DataAq} of the Preferences window. \GxsmEmph{Auto Save} is checked, each new scan is automatically saved after the scan is finished.

During image analysis it is often convenient to save the ``enhanced'' images using an easy to memorize name. Nethertheless, it is often necessary to get back to the original data. For this purpose, \GxsmEmph{Originalname} shows the name of the original data file. This feature works only for files saved using \Gxsm's NetCDF format.

The \GxsmEmph{Comment} field allows adding comments to scan data, e.g. the name of the sample. Again, saving this information is supported best for the NetCDF file format.

\GxsmHint{If your dataformat is not natively supported by \Gxsm, but can be exported to ASCII, consider using \texttt{ncdump} and \texttt{ncgen} to create a nc-file from your data. Running ncdump on any \Gxsm-nc-file shows you which parameters are necessary, insert your data in the ASCII output and revert the ASCII-file back to NetCDF using ncgen.}

%\section{Understanding the main window's entries: SPA--LEED mode}
%\label{sec:main:spaentries}
%
%The SPA--LEED version of the main window is very similar the the SPM variant. There are some changes and additional entries: The SPM
%\GxsmEmph{Range} entry is replaced by the SPA--LEED typical parameter scan \GxsmEmph{Lenght}. There is no \GxsmEmph{Steps} entry, it is always calculated from \GxsmEmph{Lenght}/\GxsmEmph{Points} and therefore the panel on right holds now two set of radio buttons to
%switch between Volts/\%BZ for length inputs and eV/Phase (S) for to energy input.
%
%In addition the SPA--LEED GUI adds inputs for \GxsmEmph{Gatetime} and \GxsmEmph{Energy}. For easy setting of the displayed (mapped) dynamic range there are entries for \GxsmEmph{CPS high} and \GxsmEmph{CPS low}.
%
%The SPA-LEED GUI offers to the already from SPM known \GxsmEmph{Offset} an additional Offset \GxsmEmph{Offset (0,0)} which
%becomes handy, if you need an Offset to center the diffraction pattern, e.g. the (0,0) spot, to the center and want to zoom in some
%higher order spots, therfore you can use the regular \GxsmEmph{Offset} starting from (0,0). Both offset are always internally added. The units of \GxsmEmph{Offset (0,0)} is always Volt, while the regular \GxsmEmph{Offset} switches between Volt/BZ consisten with
%\GxsmEmph{Length}. 
%
%This especially becomes useful using remote control programming -- see section~\ref{pi:pyremote}.
%
%\section{Printing large numbers of scans}
%\index{Printing}
%\index{Icons|see{Make Icons}}
%\index{Make Icons}
%\label{sec:main:make-icons}
%
%\GxsmMenu{Tools/Make Icons} allows to print-out series of images for browsing and quick survey. You enter a regular expression 
%in \GxsmEmph{Mask} and a path in \GxsmEmph{Path} and all files fitting into this expression are printed into a postscript-file \GxsmEmph{Icon File} for later (or immediate) printing. If the \GxsmEmph{Icon File} exists already, the new images are
%simply appended to it.
%
%Gxsm can produce these icon pages in several layouts. You may select \GxsmEmph{2x3}, \GxsmEmph{4x6} or \GxsmEmph{6x9} images per page resulting in less and larger or more and bigger icons of your scans.  The image resolution is scaled to printer-appropriate values to
%decrease render time for the printer and to minimize to risk of memory shortage for your page-printer.
%
%A number of options can be selected from pull-down menus to improve your print-out. Play with the possibilities to find your favorite combination.  The avaible options are described in Tab \ref{tab:icons:opt}.
%
%\begin{table}
%\begin{tabular}{l|l|l}
%option  & value & description\\\hline\hline
%Paper & & Papersize\\
% & A4/letter & Switch between DIN A4 and US letter\\
%\hline
%Resolution & & Optimize the output for given printer \\
% & 300/600/1200dpi & Printer resolution\\
%\hline
%E Regression & & Plane removal\\
% & no & Off\\
% & E 30/5\% & The outer 30 or 5\% of the scan are\\
% &          & omitted for plane regression\\
%\hline
%L Regression & & Background substraction\\
% & no & Off\\
% & lin.\ Reg. & Linear background from each line\\
%\hline
%View Mode & & Same as in main window\\
%\hline
%Auto Scaling & & Contrast and brightness calculation\\
% & default & uses stored settings (as set at time of save)\\
% & auto 5/20/30\% Rand & Omitting outer scan area for calculation\\
%\hline
%Scaling & & additional bitmap scaling\\
% & min-max & scale to full 8 bits\\
% & cps-lo-hi & use CPSHi/Low (SPA-LEED)\\
%\end{tabular}
%\caption{Options for the \GxsmMenu{Tools/Make Icons} function.}
%\label{tab:icons:opt}
%\end{table}

\section{Drag and Drop}
\label{sec:main:DnD}

Gxsm accepts all loadable files via 'drag and drop', e.g. from the Nautilus and understands VFS file paths.  Even
dragging URL's pointing to loadable files on the web is possible.

If you drop a file on a channel-window, it is loaded into that channel. To create a new window with a new channel bound to it, drop the file above the main window.

\section{Keyboard-Accelerators}
\label{sec:main:accel}

Most common used action on a scan-view are assigned to keyboard accelerators, this is indicated by the Key-Symbol on the right side of a menu entry. (See pull down/pop-up menus on scan 2D view). F2 for example triggers a auto-display (auto scale to min-max of all data or via active rectangle area selection).


\chapter{Channels}
\label{ch:channels}
\index{Channels}

One of the most important features of Gxsm is the
multichannel-capability.  Multichannel-capability describes the simultaneous data acquisition
and display from different sources. You may e.g.\  at the same time
measure topography and friction-forces with the AFM.  Additionally to the
simultaneous acquisition of different signals, the multi-channel feature of
\Gxsm\ can be used to load multiple images and, e.g.\ compare them, or apply
more complex operations to them.  Furthermore, it serves as a history
mechanism during image manipulation, because the result of any mathematical
operation on one channel does not overwrite it but is stored in a new channel.

\GxsmScreenShot{channelselector}{Channelselector Window.}

\section{The channel dialog}
\label{sec:channels:dialog}
%\begin{figure}[hbt]
%\center \fighalf{ChannelSelector}
%\caption{Channel Selector Window}
%\label{fig:ChannelSelectorWindow}
%\end{figure}

The channel dialog pops up selecting \menuentry{Ch. Sel.} in the
\menuentry{Windows} menu of the main window.
You can use it, to select the displaymode (\menuentry{View}) and the source for data
acquisition (\menuentry{Mode}) for any channel.

One of the following modes can be chosen, see \ref{Gxsm-Visualisation} for details:
\begin{enumerate}
\item \menuentry{No:} During data acquisition no data is displayed -- background storage/saving only. You can switch view mode any time.
\item \menuentry{Grey 2D:}  The data are displayed as a grayscaled/false-color image.
\item \menuentry{Surface 3D:}  Three/multi dimensional data/scan viewer. OpenGL (4.0 minimum) based. Can display volume data and slices of multilayered data.
\item \menuentry{Profile 1D:}  Profile view of the current or all lines. 
\end{enumerate}
Usually you will use the mode \menuentry{Grey 2D} for data acquisition. If you want to see the line profiles of the actual scan line, right-click on the window of the channel, select \menuentry{view} and activate \menuentry{red Profile}.
\index{Channel modes}

For processing of data several modes are available:
\begin{enumerate}
\item \menuentry{Off} or \menuentry{On}. Off deletes the channel.
\item \menuentry{Active}. The most important mode, it sets the active channel. All image manipulation is done using this channel. Only one channel can be \menuentry{Active} at a time.
\item \menuentry{Math} Channel, which stores the result of the last operation is automatically called \menuentry{Math}.
\item \menuentry{X} Needed for several math/image-manipulation, that need more than one source.
\end{enumerate}

If you want to activate a channel for data acquisition, please select in the second column the channel name, i.e. \menuentry{Topo} or \menuentry{ADC0\_ITunnel}. This channel will be used as a target for a scan. The toggle \menuentry{$->$} and \menuentry{$<-$} in the third column defines the scanning direction at which data are collected. Thus \filename{+Topo} means measurement of your topography during movement of your scanhead in +X direction. By chosing \menuentry{$2>$} or \menuentry{$<2$} an \emph{experimental} mode is activated in which each scanline is scanned twice, i.e. this mode is used for magnetic force measurements.

\GxsmHint{Using the MK3-A810 DSP with flexible signal configuration 4 special modes can be configured to acquire any available signal. See DSP-Control.}

\GxsmHint{The names of the input channels can be customized (see \ref{ch:config}).}

%%% Local Variables: 
%%% mode: latex
%%% TeX-master: "Gxsm-main"
%%% End: 
