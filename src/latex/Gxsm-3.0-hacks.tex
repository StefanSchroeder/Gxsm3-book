%/* Gxsm-3.0 - Gnome X Scanning Microscopy
% * universal STM/AFM/SARLS/SPALEED/... controlling and
% * data analysis software: Documentation
% * 
% * This software is based on the earlier documentation of XSM, GXSM and GXSM2
% * 
% * Copyright (C) 2017 Percy Zahl, Thorsten Wagner
% *
% * Authors: Percy Zahl <zahl@users.sf.net>, Andreas Klust <klust@users.sf.net>, Thorsten Wagner <stm@users.sf.net>
% * WWW Home: http://gxsm.sf.net
% *
% * This program is free software: you can redistribute it and/or modify
% * it under the terms of the GNU General Public License as published by
% * the Free Software Foundation, either version 3 of the License, or
% * (at your option) any later version.
% *    
% * This program is distributed in the hope that it will be useful,
% * but WITHOUT ANY WARRANTY; without even the implied warranty of
% * MERCHANTABILITY or FITNESS FOR A PARTICULAR PURPOSE.  See the
% * GNU General Public License for more details.
% *   
% * You should have received a copy of the GNU General Public License
% * along with this program.  If not, see <http://www.gnu.org/licenses/>.
% */

\chapter{\Gxsm\ Tipps and Tricks}
\label{cha:Gxsm-hacks}
\index{Tipps and Tricks}
\section{Running Two Instances of \Gxsm\ }
Because of \Gxsm\ and glib's new application management, any attempt to start a new "instance" on the same user session/account will NOT span a new process per default, but connect to an eventually existing \Gxsm process -- no matter if running with hardware connected or not.

This is unfortunate in case that you want a independent process for just reviewing data on the same login/X11 display while running \Gxsm\ in data aquisition mode the same time.

To circumvent this complication create a new user account (here simply named "gxsm@localhost") for analysis and run a \GxsmCmd{ssh -X} (with auto X forwarding) session on your current account while \Gxsm\ may be running with hardware connection on "this localhost" like shown below. Here "user@spm" is the actual user and machine running the X11 session on this machine with hardware.

If no alternative account exists, simply run as root \GxsmCmd{adduser gxsm} before.

Terminal output/sesper defaultsion example:\\ \\
\textbf{user@spm}\GxsmCmd{ssh -X gxsm@localhost}\\ \\
Within this new session/terminal you can safely start a new instance of \Gxsm .\\ \\
\textbf{gxsm@localhost}\GxsmCmd{gxsm3 -h no}\\ \\
.... now you have a new \Gxsm\ application running and it will also accept drag and drop from nautilus while in no means disturbing any existing gxsm3 process :)
