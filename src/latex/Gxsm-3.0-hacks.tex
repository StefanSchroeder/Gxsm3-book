%/* Gxsm-3.0 - Gnome X Scanning Microscopy
% * universal STM/AFM/SARLS/SPALEED/... controlling and
% * data analysis software: Documentation
% * 
% * This software is based on the earlier documentation of XSM, GXSM and GXSM2
% * 
% * Copyright (C) 2017 Percy Zahl, Thorsten Wagner
% *
% * Authors: Percy Zahl <zahl@users.sf.net>, Andreas Klust <klust@users.sf.net>, Thorsten Wagner <stm@users.sf.net>
% * WWW Home: http://gxsm.sf.net
% *
% * This program is free software: you can redistribute it and/or modify
% * it under the terms of the GNU General Public License as published by
% * the Free Software Foundation, either version 3 of the License, or
% * (at your option) any later version.
% *    
% * This program is distributed in the hope that it will be useful,
% * but WITHOUT ANY WARRANTY; without even the implied warranty of
% * MERCHANTABILITY or FITNESS FOR A PARTICULAR PURPOSE.  See the
% * GNU General Public License for more details.
% *   
% * You should have received a copy of the GNU General Public License
% * along with this program.  If not, see <http://www.gnu.org/licenses/>.
% */

\chapter{\Gxsm\ Tips and Tricks}
\label{cha:Gxsm-hacks}
\index{Tips and Tricks}
\section{Running Two Instances of \Gxsm}

\begin{description}
  \item[Problem] \mbox{}\\ You want to run a second instance of
	\Gxsm{} without interfering with a running measurement.
  \item[Underlying issue] \mbox{}\\ 
	Because of \Gxsm's and glib's new application management, any attempt
	to start a new instance on the same user session/account will
	\emph{not} spawn a new process per default, but connect to an existing
	\Gxsm{} process -- no matter if the system is running with hardware
	connected or not.

	This is unfortunate in case that you want an independent process for
	just reviewing data on the same login/X11 display while running \Gxsm\
	in data aquisition mode at the same time.
  \item[Solution] \mbox{}\\ 
\begin{tabular}{ll}
	Step 1 & Create a new user account. As root run \texttt{\#adduser alice}\\
	Step 2 & Setup a password for the new user.\\
	Step 3 & Run as the new user: \texttt{\$ssh -X alice@localhost gxsm3 -h no} \\
	Note: & SSHD must be configured to allow X11-forwarding.
\end{tabular}
\end{description}

%To circumvent this complication, create a new user account (which we
%will call ``gxsm@localhost'' for analysis and run a \GxsmCmd{ssh -X} (with
%auto X forwarding) session on your current account while \Gxsm\ may be
%running with hardware connection on ``this localhost'' like shown below.
%Here ``user@spm'' is the current user and machine running the X11 session
%on this machine with hardware.
%
%If no alternative account exists, simply run as root \GxsmCmd{adduser gxsm} before.
%
%Terminal output/sesper defaultsion example:\\ \\
%\textbf{user@spm}\GxsmCmd{ssh -X gxsm@localhost}\\ \\ Within this new
%session/terminal you can safely start a new instance of \Gxsm .\\ \\
%\textbf{gxsm@localhost}\GxsmCmd{gxsm3 -h no}\\ 
%
%Now you have a new \Gxsm\ application running and it will also accept
%drag and drop from nautilus while in no means disturbing any existing
%\Gxsm\ process :)
