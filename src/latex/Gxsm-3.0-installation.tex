%/* Gxsm-3.0 - Gnome X Scanning Microscopy
% * universal STM/AFM/SARLS/SPALEED/... controlling and
% * data analysis software: Documentation
% * 
% * This software is based on the earlier documentation of XSM, GXSM and GXSM2
% * 
% * Copyright (C) 2017 Percy Zahl, Thorsten Wagner
% *
% * Authors: Percy Zahl <zahl@users.sf.net>, Andreas Klust <klust@users.sf.net>, Thorsten Wagner <stm@users.sf.net>
% * WWW Home: http://gxsm.sf.net
% *
% * This program is free software: you can redistribute it and/or modify
% * it under the terms of the GNU General Public License as published by
% * the Free Software Foundation, either version 3 of the License, or
% * (at your option) any later version.
% *    
% * This program is distributed in the hope that it will be useful,
% * but WITHOUT ANY WARRANTY; without even the implied warranty of
% * MERCHANTABILITY or FITNESS FOR A PARTICULAR PURPOSE.  See the
% * GNU General Public License for more details.
% *   
% * You should have received a copy of the GNU General Public License
% * along with this program.  If not, see <http://www.gnu.org/licenses/>.
% */

\chapter{\Gxsm\ Project Installation}
\label{cha:Gxsm-install}
\index{Installation}

\section{System requirements}
\index{System requirements} 
\Gxsm\ needs a reasonably up-to-date machine running a recent version of almost
any Linux variant. \Gxsm\ is always developed on a Debian Testing. Some
description to install \Gxsm\ an a clean Debian Jessie are found in this forum
posting: \GxsmWWWs{sourceforge.net/p/gxsm/discussion/297458/thread/3f0faafe/}
It also runs on Ubuntu 20.04 LTS for which dedicated binary packages are available. 

There are many more forum entries related to the installation processes on 
different linux distributions.

System memory requirements are ranging from little to several gigabytes if you
want to deal with big scans, movies or multidimensional data sets \smiley.

\GxsmNote{The \Gxsm\ Windows port supports SRanger based hardware. Binaries are
available via a separate project hosted on sourceforrge.net:\\
\GxsmWWWs{sourceforge.net/projects/gxsm4w/}.}

\GxsmWarning{The GXSM for Windows project was not updated since 2013.}

\ \\

To install/compile \Gxsm\ on Debian Testing install these package:

\begin{verbatim*}
apt-get install subversion libgnomeui-dev intltool yelp-tools
 gtk-doc-tools gnome-common libgail-3-dev libnetcdf-dev
 libnetcdf-cxx-legacy-dev libfftw3-dev libgtk-3-0
 libgtksourceview-3.0-dev gsettings-desktop-schemas-dev
 python-gobject-2-dev libgtksourceviewmm-3.0-dev
 libquicktime-dev libglew-dev freeglut3-dev libgl1-mesa-dev
 libopencv-core-dev libopencv-features2d-dev libopencv-highgui-dev
 libopencv-objdetect-dev libnlopt-dev libglm-dev fonts-freefont-ttf
\end{verbatim*}


\GxsmNote{gtk+-3.0 $>=$3.22.0 is not included in Ubuntu 16.04 LTS. If you want
to compile \Gxsm\ with prior versions of gtk3, you have to patch the file configure.ac.}


\section{Getting \Gxsm\ from SVN}
\index{Download|see{SVN!checkout}}

Go to the \Gxsm\ Project's home page at
\GxsmWWW{sourceforge.net/projects/gxsm} and follow the link almost at the
bottom ``SVN Repository'' and have a look at the instructions there. 

The ``module'' contains the C++ source code for ``Gxsm-3.0''. It can be retrieved with subversion
anonymously:


\GxsmCmd{\small svn co svn://svn.code.sf.net/p/gxsm/svn/trunk/Gxsm-3.0 Gxsm-3.0}


The module ``Gxsm-3.0-Manual'' can be checked out, too. It contains the
\LaTeX-sources for the manual that you are currently reading.


\GxsmCmd{\small svn co svn://svn.code.sf.net/p/gxsm/svn/trunk/Gxsm-3.0-Manual Gxsm-3.0-Manual}


\subsection{Configuration}
\index{configure}
\index{autogen.sh}

Change into the \GxsmFile{gxsm3-svn} directory and type

\GxsmCmd{./autogen.sh}\\[1mm]
This should finally create all Makefiles in several sub-directories.

\subsection{Compilation}
\index{compiling}
Type make in the \GxsmDir{gxsm3-svn} directory:

\index{make}
\GxsmCmd{make}\\[1mm]
This command will compile everything, if you want to compile it step by step,
you can run this command in the \GxsmDir{src} sub-directory first separately to
see if all works OK. But later you want to compile all plug-ins and tools as
well -- if it works in \GxsmDir{src} it is most likely it works OK for the
rest, so try make in the \GxsmDir{gxsm-svn} dir again!

\GxsmHint{To speed the make process up on multi core systems, use \GxsmCmd{make
-j2}, \GxsmCmd{make -j4}\dots for spawning multiple compile jobs.}

\subsection{Installation}
\index{Installation}
\index{make install}
Finally you need to run the install command:

\GxsmCmd{[sudo] make install} \hfill (root privileges needed!)

If there are any errors, in most cases there are missing development packages
(e.g. missing header files/libraries), have a look at the first occurring error
and figure out what's missing, refer to \ref{sec:Gxsm-install-troubleshooting}.

\GxsmHint{For automatic thumb-nailing within the Nautilus file manager and
setting up other goodies, check out the \Gxsm\ \GxsmEmph{Tips and tricks}
forum:\\
\GxsmWWWx{sourceforge.net/forum/forum.php?forum\_id=302195}{http://sourceforge.net/forum/forum.php?forum_id=302195}}

\subsection{Updating}
\index{update|see{SVN}}
\index{SVN!update}

For later updates just run

\GxsmCmd{svn update}\\[1mm]
in the \GxsmDir{gxsm3-svn} directory and remake/install all again!

If you want to keep your previous binary version/installation, use a different
installation path (prefix):

\index{prefix}
\GxsmCmd{./configure $--$prefix=/usr/local/gxsm3-test ; make; make install}\\[1mm]

\subsection{Trouble shooting}
\index{troubleshooting}
\label{sec:Gxsm-install-troubleshooting}

If your configuration or compilation fails, most likely a dependency is missing
and needs to be installed before trying again. We try hard to keep the SVN
source tree in a sane state, one which should always allow an error free
compilation. All developers are strongly encouraged to adhere to this
golden-rule of SVN usage -- and as far as I remember the occurrences of broken
check-ins are very close to zero and for sure of a very short lasting.
So if you experience errors, they are unlikely to be found in the source
itself, but rather than in the development environment. A known
issue is the fact, that the current \Gxsm\ configuration script does not cover
all checks and error reports if a needed include file or library is missing,
thus you will get to the point to start the compilation and then missing
include files may be reported.

However, this does not mean it's all \GxsmEmph{bug-free}. If you are
experiencing runtime bugs or crashes, please report them here:\\
\GxsmWWWx{sourceforge.net/tracker/?group\_id=12992\&atid=112992}{http://sourceforge.net/tracker/?group_id=12992&atid=112992}

The following list may help a bit to locate/solve trouble at
compilation/installation time:

\begin{description}
\item[configuration trouble] if the \GxsmFile{./autogen.sh} fails:\\
 Check the long output for error/missing messages. Check if you have all
necessary libraries and development files installed! Especially the
\GxsmFile{gnome-autogen.sh} script must be available.

\item[compilation/make/cc] Assuming the \GxsmFile{./autogen.sh} finished with
the creation of all Makefiles: At least the first gcc call should be
started\dots Then the most likely errors are generated by missing include
files, check for the first occurring error/missing includes!

Second, have a look if the files, that are noted missing in your error messages
are really absent or just installed in unusual places.

\item[compilation/linking] If this fails, something is messed up with your
development packages and library installation, usually unlikely to happen. Make
sure there is no version mismatch between development/include files and
libraries installed/configured!

\item[installation] Very unlikely this fails, except your permissions are not
sufficient (not root?) or you are running out of disk space (check with df -h).

\item[call gxsm3] Please do not mix up different installations of \Gxsm. The
installation from SVN and from the APT repository usually use
different folders to store the plug-ins and so on. If you intend to use the
latest version from the SVN repository, deinstall your deb-packages

\GxsmCmd{[sudo] apt-get purge gxsm3}\\[1mm]
\end{description}

Also, check the online \Gxsm-Forum
\GxsmWWWs{sourceforge.net/forum/?group\_id=12992} for additional information
and reports by other users! If you need further assistance/help, please make
use of the \GxsmEmph{Help} forum there!

\section{Getting \Gxsm\ via APT}
If you want to contribute to the \Gxsm-project, we strongly recommend to use
the latest source code available via the SVN repositry on sourceforge.net. But
for just using \Gxsm\ (on an Ubuntu based system) using your package manager
APT is the easier alternative.

\subsection{Adding package repositories}
First you have to add the PPA (Personal Package Archive) hosted on launchpad:

\GxsmCmd{[sudo] apt-add-repositry ppa:totto/gxsm}

\GxsmNote{On Ubuntu 16.04 LTS you will need to add also the PPA from the
Gnome3-team to obtain recent version of the GTK+3.0 libaries.}

\GxsmCmd{[sudo] apt-add-repository ppa:gnome3-team/gnome3}

\GxsmCmd{[sudo] apt-add-repository ppa:gnome3-team/gnome3-staging}

\subsection{Package installation}
Refresh you local database:

\GxsmCmd{[sudo] apt update}\\[1mm] and install the \Gxsm\ package via

\GxsmCmd{[sudo] apt install gxsm3 [sranger-modules-std-dkms
sranger-modules-mk23-dkms]}\\[1mm] The last command will not only install
\Gxsm\ (and the optional sranger packagesin the [] brackets) but also fullfil
the required dependencies. 

Once you have installed \Gxsm\ you will be noticed by your package manager (in
the lastest Ubuntu version this is called Ubuntu Software Center) about
updates.

\GxsmWarning{In any case, please do not mix up SVN and APT installation.} 

\subsection{Updating}

No further action is required. The package manager usually checks frequently
for new versions of the \Gxsm\ package and will offer to update it.

\subsection{Building packages} 

The SVN also contains all files to make your own deb-package. To do so
checkout the source via subversion as described above. Then enter the folder
gxsm3-svn and the subfolder debian (assuming that you checked out the
source code in gxsm3-svn) and rename control\_$<$your ubuntu$>$ to control.

Then you can use 

\GxsmCmd{dch -i}

\GxsmCmd{dch -r}

to update the changelog file of the package. In particular, you might want to
increase the version number. Then use

\GxsmCmd{debuild -b -I -uc -us}

to make your (unsigned) package. To make a signed package use

\GxsmCmd{debuild -b -I -k$<$your email$>$}

The last command will require the variables DEBEMAIL and DEBFULLNAME to be set, i.e., you may want to add to your .bashrc\\
DEBEMAIL="$<$your email$>$"\\
DEBFULLNAME="$<$your full name$>$"\\
export DEBEMAIL DEBFULLNAME\\
\\
Package built this way can be installed via \GxsmCmd{sudo dpkg -i $<$package-name$>$.}
 
%%% Local Variables: 
%%% mode: latex
%%% TeX-master: "Gxsm-main"
%%% End: 

