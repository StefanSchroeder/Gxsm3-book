%/* Gxsm - Gnome X Scanning Microscopy
% * universal STM/AFM/SARLS/SPALEED/... controlling and
% * data analysis software: Documentation
% * 
% * Copyright (C) 1999,2000,2001 Percy Zahl
% *
% * Authors: Percy Zahl <zahl@users.sf.net>
% * additional features: Andreas Klust <klust@users.sf.net>
% * WWW Home: http://gxsm.sf.net
% *
% * This program is free software; you can redistribute it and/or modify
% * it under the terms of the GNU General Public License as published by
% * the Free Software Foundation; either version 2 of the License, or
% * (at your option) any later version.
% *
% * This program is distributed in the hope that it will be useful,
% * but WITHOUT ANY WARRANTY; without even the implied warranty of
% * MERCHANTABILITY or FITNESS FOR A PARTICULAR PURPOSE. See the
% * GNU General Public License for more details.
% *
% * You should have received a copy of the GNU General Public License
% * along with this program; if not, write to the Free Software
% * Foundation, Inc., 59 Temple Place, Suite 330, Boston, MA 02111-1307, USA.
% */


\chapter{Front Matter}
\label{ch:frontmatter}

\large\textbf{
 The Gxsm Project: Gxsm itself, drivers, utilities, demos and documentation\\[2mm]
 Copyright (C) 1999 - 2017 Percy Zahl, Andreas Klust, et al\\
 Email: \GxsmEmail{zahl@users.sourceforge.net}}
\large\textbf{WWW: \GxsmWWW{gxsm.sourceforge.net}}\\


 This program is free software; you can redistribute it and/or modify
 it under the terms of the GNU General Public License as published by
 the Free Software Foundation; either version 2 of the License, or
 (at your option) any later version.

 This program is distributed in the hope that it will be useful,
 but WITHOUT ANY WARRANTY; without even the implied warranty of
 MERCHANTABILITY or FITNESS FOR A PARTICULAR PURPOSE. See the
 GNU General Public License for more details.

 You should have received a copy of the GNU General Public License
 along with this program; see the file COPYING. If not, write to
 the Free Software Foundation, Inc., 59 Temple Place - Suite 330,
 Boston, MA 02111-1307, USA.

 See the end of this document for complete license: Appendix~\ref{app:gpl}

\newpage

\chapter{Introduction}
\marginpar{\includegraphics[width=1in]{images/GXSM-penguin}}

\label{ch:intro}

\Gxsm\footnote{The Project can be found in the Internet at
\GxsmWWW{gxsm.sourceforge.net}} -- Gnome X Scanning Microscopy -- is
a powerful graphical interface for any kind of 2D and 3D (multi layered
2D mode) data acquisition methods, especially designed for scanning
probe microscopy (SPM).

\GxsmScreenShot{gxsm3-main}{Gxsm3 (version 3.47.0 Action Eclipse) Main Scan Control Window.}

The aim of the \Gxsm-project is to provide a versatile control system being 
suitable to operate all different kinds of scanning probe microscopes. This 
includes in particular scanning tunneling microscopes (STMs) and atomic force 
microscopes (AFMs), but it is not restricted to these types. In principle, it
is also flexible enough to operate scanning angle resolved light scattering 
(SARLS) experiments or spot profile analysis of low-energy electron diffraction
(SPA-LEED) optics. One reason for this versatility is that all these instrument
have in common a 2D-data acquisition by scanning sequentially points in the 
$xy$-plane.

Certainly, \Gxsm\ gains also versatility, because of its design concept: a DSP
(digital signal processor) based hardware is used for the data acquisition, the
scan signal generation and various feedback lopes ($z$ distance between tip 
and sample but also the oscillation control of an NC-AFM). The graphical user 
interface (GUI) provides not only tools for 1D (line profiles), 2D (images), 3D 
(morphology), and even 4D (time series) data visualization, but also for
manipulation and analysis. As the GUI is based on plug-ins, it can be easily
extended to new tasks. 

A third aspect is the python interface of \Gxsm\ allowing to control the GUI
remotely at almost real-time level. The performance of the python scripting is
more than sufficient to provide a functionality far beyond batch acquisition
and processing of data, but also allowing to code complex data aquisition tasks
without programming anything directly on DSP level.

The project was founded at the Institute for Solid State
Physics\footnote{Institut f�r Festk�rperphysik, Universit�t Hannover,
 Appelstra�e 2, D-30167 Hannover, Germany\\
 www.fkp.uni-hannover.de} Leading developer of \Gxsm\ is Percy
Zahl\footnote{e-mail: zahl@users.sourceforge.net}, but more than 50 SPM groups
world wide are not just using \Gxsm\ but also contribute to its further
development. The program was developed for Linux\footnote{For example Debian 
9.3} using the Gtk$+$/Gnome libraries\footnote{\Gxsm\ currently requires Gnome 
3.2 and GTK+3.22} for the GUI.
 
And best of all: it's free! \Gxsm\ is licensed under the terms of the
GNU General Public License (GPL, see \ref{app:gpl}). Therefore,
everyone can copy, use, and modify \Gxsm\ for his/her needs provided
that the resulting software is again published under the GPL licence.

The following list gives a short overview on the main features of \Gxsm:
\begin{itemize}
\item Support for STM, AFM, and any other 2D (2D layered and multi-channel) 
data-acquisition method can be supported by \Gxsm.

\item The \Gxsm\ core handles multiple channels of 2D (layered) data fields of
arbitrary type and unlimited size and a grey-scale or false color view of 2D 
data image, using ''on the fly'' data transformation as there are: 
% could be removed here as there is a whole chapter about the visualisation!!! 
\begin{itemize}
 \item ''Quick'': a line regression and subtraction is performed on
 each scan-line
 \item ''Direct'': only contrast and brightness adjustments 
 \item ''Direct HiLit'': same as direct but marks the lowest and highest values
 \item ''Plane'': on-the-fly 3-point defined plane subtraction
 \item ''Logarithmic'': logarithmic scaling mode, almost used by
 diffraction methods
 \item ''Horizontal'': automatic line average subtraction
 \item ''Differential'': view of X-derivative, $[Z(X+1)-Z(X)]$
 \item ''Periodic'': like ''Direct'' Mode, but grey-scale is applied
 modulo \#grey-levels
 \item More sophisticated background correction and data analysis
 methods are implemented as filters.
 \end{itemize}
For more details on the different modes of the visualization of the data see Chap.~\ref{ch:visual}.

\item Data presentation is by default a (grey or false color) image
 but it can be switched to a profile view (1d), profile extraction
 on the fly... Or you can use a 3D shaded view (using OpenGL 4.6) which
 now offers a sophisticated scene setup.

\item The ''high-level'' scan controller is now separated from the \Gxsm\
 core and is build as PlugIn, while the real-time ''low-level''
 scanning process, data-acquisition and feedback loop (if needed),
 runs on the DSP -- if present, else a dummy image is produced. The
 current scan-line, marked in red, can be viewed simultaneously as
 profile. (View$\rightarrow$red Profile)
 
\item Extremely flexible configuration of user settings and data
 acquisition and probe modes.

 \begin{itemize}
 \item Easy to extend by Plug-ins, some examples of existing
 Plug-ins:
 \begin{itemize}
 \item Background correction methods
 \item Image filtering 1D and 2D, including several methods for Fourier
 transformation 
 \item Image analysis/statistics: histogram, step analysis, ...
 \item Geometric transformations: scaling, rotation, affine, ...
 \item and more, write and contribute your favorite math Plug-in
 for \Gxsm! Don't be afraid, there is a step by step
 instruction tutorial and a math Plug-in generator, all you
 need to do is to add your math code!
 \end{itemize}

 \item Special datafile/formats import/export filters, (extendable via PlugIns):
 \begin{itemize}
 \item a set of simple raw formats (.byt,.sht,.flt,...), see \ref{pi:primitiveimexport}
 \item Digital Instruments/Veeco Metrology Group, NanoScope (import)
(go to \GxsmWWW{veeco.com} to learn more, see \ref{pi:nanoimport})
 \item Omicron NanoTechnology, Scala (\GxsmWWW{www.omicron.de}, see \ref{pi:OmicronIO})
 \item WSxM/Nanotec Electronica SPM, WSxM (\GxsmWWW{www.nanotec.es}, see \ref{pi:WSxMio})
 \item SDF (Surface Data Format, see \ref{pi:sdfimport})
 \item UK2000 v3.4, see \ref{pi:UK2kimport}
 \item G. Meyer STMAFM, see \ref{pi:gmeyerimexport}
 \item Park Scientific (AFM, basic import support, see \ref{pi:primitiveimexport})
 \item ``d2d'' data format (SPA--LEED), used by ``spa4'' and ``xspa'', see \ref{pi:spa4imexport}
 \item any NetCDF file containing a 2D data array
 \item Grey Images in .pgm format (P5 type), see \ref{pi:pngImExport}
 \item PNG image format, see \ref{pi:pngImExport}
 \item Targa (.tga) export, in 8/16/24bits color depth, \ref{pi:primitiveimexport}
 \end{itemize}

 \item Special instrument control Plug-Ins:
 \begin{itemize}
 \item a Scan Control Panel
 \item SPM: DSP Control: Feedback and Scan Characteristics
 (Speed, ...)
 \item SPM: Mover/slider and auto approach controls
 \item SPM: flexible Probing: Spectroscopy (STS), Force-Distance
 Curves in AFM, using the DSP also Digital LockIn Probing (e.g.
 dI/dU (U)) is possible without any additional hardware! But
 not only SPM, I already ran our Quadrupole Mass Analyzer with
 it!
 \item Phase Lock Loop (PLL) operation in combination with the SR-MK3Pro \& A810 boards (with enabled PLL option)
% \item SPA-LEED: Peak Finder, Focus-Tool, Rate-Meter, Peak
% Intensity Monitor (used with SPA-LEED test module)
% \item SPA-LEED Control (used with SPA-LEED test module)
% \item CCD control (very special, just for gurus)
	
 \end{itemize}

 \item NanoPlotter Plug-in: reads simple HPGL files and moves along the
 plot path using predefined DSP settings (U, I, Feedback Parameters,
 Speed, ...) for ''Pen-Up'' and ''Pen-Down'' movements. This was in
 principle already possible via the remote control and a script, but
 now it's much more convenient and user friendly! Even the path is
 shown ... can be modified, saved and re-plotted!

 \item A Plug-in categorization mechanism automatically only loads the
 Plug-ins for the actual setup: E.g. no Hardware Control
 Plug-ins are loaded in ``offline'' Data Analysis Mode.

 \item At the time there are more than 80 Plug-ins.
 \end{itemize}

\item \Gxsm\ itself is fully hardware independent. It provides a
 generic hardware-interface (HwI) plugin infrastructure to attach
 any kind of hardware. The HwI has to manage low level tasks and
 pass the data to the GXSM core, plus, it has to provide the
 necessary GUI to provide user access and control to
 hardware/instrument specific parameters and tasks.
 
\item Scan parameter changing on the fly -- you can modify the
 feedback parameters or switch the tunneling voltage
 while scanning in between scan lines. For example: Imagine you are
 scanning a STM topography and current Image, the surface looks flat,
 then just change the feedback parameters to CP=0 and CI=1e-5
 (something small) and now you are in constant height mode!
 
\item On the fly, even while scanning is in progress, you can view
 profiles, extract data parts, re-scale -- just do all you like!
 
\item Event mechanism: User ``Events'' like bias change are now logged
 and attached to the scan data and can be visualized (Position
 marker). Other events like ``Probe'' (any kind of spectroscopy
 or manipulation) while or after scanning are automatically
 attached to the scan with all data and can be visualized via a
 single mouse click. The same for automatic rastered ``Probe'' scanning.

\item Python Remote Control Interface: The \Gxsm\ scanning progress is 
 scriptable using Python. Or may be used to automatize math/data
 analysis tasks.
 
\item Cross Platform: works on i386 and PPC based Linux distributions; a Windows port besides the main stream is available\footnote{\GxsmWWWx{gxsm4w.sf.net}{http://gxsm4w.sf.net}}.
 
\item \Gxsm\ takes full advantage of the NetCDF data format.
 
\item Scan auto saving, session logging, Plug-in details browser,
 NC-View, PS-Printing and a Icon generator are available too.
 
%\item Included Add Ons:
%\item Linux COFF loader to launch the DSP program on the SRanger hardware (old:
% PC31, PCI32 cards are not any longer GXSM main stream).
% See the SRanger Linux project\footnote{included in the SRanger project related
% to Gxsm, refer to \GxsmWWW{sranger.sourceforge.net}}.
%\item the required kernel modules
%\item A universal SRanger DSP binary (precompiled, in COFF format).
%\item A set of external utilities written in Python for monitoring and
% configuring the DSP board.
\end{itemize}

The \Gxsm\ software can be divided into three parts: First, the \Gxsm\ core
providing the main functionality for handling and visualization of data
described in the first part of this manual. The basic functions of the
\Gxsm\ core can be extended using plug-ins. Plug-ins are small pieces of
software dynamically linked to the core. The plug-ins are described in the
second part of the manual. The third part documents the digital signal
processing (DSP) software needed to carry out actual measurements. The DSP
software is not necessary for applications using \Gxsm\ only for data
analysis purposes.
