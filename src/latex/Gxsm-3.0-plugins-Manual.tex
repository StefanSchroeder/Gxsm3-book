%/* Gxsm - Gnome X Scanning Microscopy
% * universal STM/AFM/SARLS/SPALEED/... controlling and
% * data analysis software:  Documentation
% * 
% * Copyright (C) 1999,2000,2001 Percy Zahl
% *
% * Authors: Percy Zahl <zahl@users.sf.net>
% * additional features: Andreas Klust <klust@users.sf.net>
% * WWW Home: http://gxsm.sf.net
% *
% * This program is free software; you can redistribute it and/or modify
% * it under the terms of the GNU General Public License as published by
% * the Free Software Foundation; either version 2 of the License, or
% * (at your option) any later version.
% *
% * This program is distributed in the hope that it will be useful,
% * but WITHOUT ANY WARRANTY; without even the implied warranty of
% * MERCHANTABILITY or FITNESS FOR A PARTICULAR PURPOSE.  See the
% * GNU General Public License for more details.
% *
% * You should have received a copy of the GNU General Public License
% * along with this program; if not, write to the Free Software
% * Foundation, Inc., 59 Temple Place, Suite 330, Boston, MA 02111-1307, USA.
% */

\chapter{Using plug-ins}
\label{ch:plugins-use}

The \Gxsm\ core provides only basic data handling and visualization
functions.  All more advanced image manipulation and analysis functions are
implemented in form of plug-ins.  A plug-in is basically a small piece of
software designed for a usually very limited purpose, e.g.\ a low pass
filter.  Plug-ins, however, don't work as standalone programs but are
dynamically linked to the \Gxsm\ core.  \Gxsm\ plug-ins usually register
themselfs to menus within \Gxsm.  Therefore, if everything works fine, the
user doesn't notice at all that she/he is using a plug-in function and not a
function hard linked to the \Gxsm\ core.

The plug-ins are automatically loaded during the start of \Gxsm.  Instead of
simply loading all avaible plug-ins, \Gxsm\ loads only the plug-ins used for
the current \GxsmEmph{Instrument type} (see Sec.\ \ref{sec:conf:instr}).
For instance, SPA-LEED related plug-ins are not loaded if \Gxsm\ is
conbfigured for STM as \GxsmEmph{Instrument type}.  During runtime, \Gxsm\
can be forced to reload the plug-ins by \GxsmMenu{Tools/Reload Plugins}.
If the \GxsmEmph{Load All} button is pressed in the upcoming pop-up window,
all avaible plug-ins are loaded regardless of the \GxsmEmph{Instrument type}.

The plug-in \GxsmTT{listplugins} (see Sec.\ \ref{plugin:listplugins})
is a ``plug-in plug-in''.  This plug-in can be called using
\GxsmMenu{Tools/Plugin Details} and lists all currently loaded
plug-ins in a window with detailed information on each of them.
Highlighting a plug-in in the list and pressing the
\GxsmButton{Details} displays all avaible information on it, such as
the path to the plug-in, a short description, the name of its author,
etc.  Furthermore, the \GxsmEmph{About} and more important the
\GxsmEmph{Configure} functions of the plug-in can be started from here
(it is further referenced as \GxsmEmph{PlugIn configurator}).

The core function \GxsmMenu{Tools/Plugin Info} displays some information on
all loaded plug-ins on \GxsmTT{stdout}.  The output is html formatted and
currently used to provide a list of all plug-ins on the \Gxsm\ website.

Plug-ins are the easiest method to extend the functionality of \Gxsm.
For the most common problem of writing data manipulation functions a
shell script included in the \Gxsm\ source code can be used to
automatically generate the source code for a new plug-in \footnote{Run
  the shell script \GxsmFile{./generate\_math\_plugin.sh} at the
  command prompt from the \GxsmFile{Gxsm/plug-ins} directory to
  generate a new math-type plug-in.}.  Only the code for the actual
manipulation of the data must be hand-written.  The newly generated
PlugIn provides already a simple math routine to demonstrate the
basics of source and destination data access methods provided by the
\Gxsm\ core. For those curious hacker guys a plain plug-in template
and step by step instructionset are available too
(\ref{Gxsm-PlugIn-Vorlage}).

For the second most common task, the import or export of data from or
to, respectively, files generated by other programs a well commented
example code exists and can be used as template for writing new
import/export plug-ins.

For all other special tasks like instrument control a deeper knowlegde
of the \Gxsm\ core and hardware abstraction methods is required and a
study of some existing PlugIns and/or the core classes is mandatory.
But feel free to use the discussion forum located at the \Gxsm\ 
project Web pages to get more help on this -- you are welcome by the
\Gxsm\ team.

