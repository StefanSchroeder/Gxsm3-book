%/* Gxsm - Gnome X Scanning Microscopy
% * universal STM/AFM/SARLS/SPALEED/... controlling and
% * data analysis software:  Documentation
% * 
% * Copyright (C) 1999,2000,2001 Percy Zahl
% *
% * Authors: Percy Zahl <zahl@users.sf.net>
% * additional features: Andreas Klust <klust@users.sf.net>
% * WWW Home: http://gxsm.sf.net
% *
% * This program is free software; you can redistribute it and/or modify
% * it under the terms of the GNU General Public License as published by
% * the Free Software Foundation; either version 2 of the License, or
% * (at your option) any later version.
% *
% * This program is distributed in the hope that it will be useful,
% * but WITHOUT ANY WARRANTY; without even the implied warranty of
% * MERCHANTABILITY or FITNESS FOR A PARTICULAR PURPOSE.  See the
% * GNU General Public License for more details.
% *
% * You should have received a copy of the GNU General Public License
% * along with this program; if not, write to the Free Software
% * Foundation, Inc., 59 Temple Place, Suite 330, Boston, MA 02111-1307, USA.
% */


\chapter{Visualization} 
\label{ch:visual}
\index{Visualisation}

As a multi purpose 2D/3D data visualization, acquisition and
manipulation system \Gxsm\ provides a set of powerful graphical
presentation modes. Available are 2D grey/false color, profile and 3D
representations. The internal objective structure of \Gxsm\ allows to
switch in between all available ``View-Modes'' on the fly and at all
times, even while scanning. In addition, on the fly data to color space
mappings are provided for viewing data.

\section{Data display modes}
\label{Gxsm-VModes}
\index{View modes|see{View}}
\index{Data display modes|see{View}}

Due to a possibly huge Z-value range and often small local signal
variation special data transformations (e.g. for mapping to
color space) are needed to visualize these features. Therefore from
\GxsmEmph{Quick}, \GxsmEmph{Direct}, \GxsmEmph{PlaneSub}, \GxsmEmph{Logarith.},
\GxsmEmph{Horizontal} \GxsmEmph{Diff} and \GxsmEmph{Periodic} view
modes can be selected via \Gxsm\ main window in ``View'' or via the
pop-up window of the data window itself. The raw data which are saved for example in the nc-files are not affected by a change of the data diplay mode.

\index{Quick|see{View}}
\index{View!Quick}
\index{Direct|see{View}}
\index{View!Direct}
\index{Direct HiLit|see{View}}
\index{View!Direct HiLit}
\index{Logarith.|see{View}}
\index{View!Logarith.}
\index{PlaneSub|see{View}}
\index{View!PlaneSub}
\index{Diff|see{View}}
\index{View!Diff}
\index{Periodic|see{View}}
\index{View!Periodic}
\index{Horizontal|see{View}}
\index{View!Horizontal}

\begin{description}
\item[\GxsmEmph{Quick}] for each line a line regression is estimated
  from a subset of points (for speed) and data is slope and offset
  corrected for visualization.

\item[\GxsmEmph{Direct}] data is displayed 'as is'. Only a linear
  transformation for shifting and scaling data into view range is
  performed using the view range ``VRange Z'' and view offset
  ``VOffset Z'' controls located context menue of the scan window.

\item[\GxsmEmph{Direct HiLit}] Same as Direct but marks the lowest and highest values.

\item[\GxsmEmph{Plane}] data is displayed after background correction
  by a bi-linear/offset function (plane-correction) defined by three
  Point-Objects. The resulting Z values are linear transformed to
  shifting and scale data into view range as it is defined by the view
  range ``VRange Z'' and view offset ``VOffset Z'' controls located in
  the context menue of the scan window or automatically calculated to fit a selected rectangular area.
  
\item[\GxsmEmph{Horizontal}] shifts the lines, so that their average is zero.

\item[\GxsmEmph{Periodic}] works like \GxsmEmph{Direct}, but the available
colors are used periodically repeating (mapping: linear Z transform
as in \GxsmEmph{Direct} modulo number of colors). 
\GxsmNote{The periodic mode is only via. pop up menu accessible.}

\item[\GxsmEmph{Logarithmic}] a logarithmic scaling can be applied to data.
  It used for very high dynamic data such as diffraction data (e.g.
  SPA--LEED or XRD) where background variation should be visible,
  while high intensity peaks are not out of display range. The
  translation happens in the following way:
\begin{displaymath}
  Grey/Colornumber \propto \log \left( 1 + \left| Data - Min \right| \times Contrast +
    Bright \right) \mbox{,}
\end{displaymath}
with $Min$ being the smallest value (omits negative values), Contrast
and Bright are calculated from VRange Z and VOffset Z to fit data into
desired Z Range window.

\item[\GxsmEmph{Differential}] displays a averaged and weightened
  X-derivative, similar to the Koehler filter
  \ref{PlugIn-F1D-Koehler}, but using a smaller averagening range (16
  pixels).
%Calculated as:
%\[Z(x=0) = 0, Z(x, \text{for} x=1,2,3,\dots N-1) = Z(x)-Z(x-1)\]

\end{description}
To change the view mode just select a \GxsmRadio{main window}{Quick /
  Direct / Plane /Logarith. / Diff.}.  These data display
modes are applyed to all visualization modes, see
\ref{Gxsm-Visualisation}.

\subsection{Scaling and shifting data view range}
\label{display-dialog}

\subsubsection{\GxsmEmph{View Range Z} and \GxsmEmph{View Offset Z}}
\label{bright-contrast}
\index{Scaling|see{View Range}}
\index{View Range}

The parameters \GxsmEmph{VRange Z} and \GxsmEmph{VOffset Z} found in Gxsm
main window are controlling the always applied linear
transformation of the data to grey or false color mapping.

\index{View Range!manual}
\index{View Range!VRange Z}
\GxsmEntry{VRange Z} sets the data Z-Range which should be mapped to full color space.

\index{View Range!VOffset Z}
\GxsmEntry{VOffset Z} sets the offset relative to averaged data range. E.g.
if you data represents a stepped surface (terraces assumed to be
horizontal aligned) and you set the \GxsmEntry{VRange Z} to approx the step
height you will get only one terrace in view range, others are pinned
at max/min color, using the VOffset you can select a terrace to be
viewed with high contrast.

\index{View Range!automatic}
Usually using the \GxsmToolbar{Autodisp} button will do the job automatically
for the whole scan or if a rectangle object \ref{Gxsm-VObjects}
is selected the selected area will be scaled automatically to full
color space available.
\GxsmEntry{VOffset Z} is always set to zero by \GxsmToolbar{Autodisp}.

\index{View Range!tolerant}
\GxsmHint{If your scan has some distorsions/spikes messing up the
  automatic min-max range, try enabling the \GxsmMenu{View/Tolerant
    Auto Display}-option. This will compute the view-range via an
  automatic histogram analysis in a progressive way.}

%\begin{displaymath}
%  \text{Color Number} = \text{Z-value} \times Contrast + Bright
%\end{displaymath}

\subsubsection{In SPA--LEED mode: CPS High -- Low}
\label{CPShi-lo}
\index{SPA--LEED}
\index{SPA--LEED!CPS high/low}

In SPA--LEED mode the scaling of CPS-High/Low can be set.

\subsection{Using a palette: false color mapping}

\subsection{Custom and \Gxsm\ provided palette}
\label{color-custom}
The color palette is simply a one dimensional PNM-bitmap file, that
can be selected through an entry in the preferences dialog, call
\GxsmMenu{Settings/Preferences}, and is located there
\GxsmPref{User}{User/Palette}.  Activate \GxsmMenu{View/Palette}
to make use of the selected palette and refresh the view via 
the \Gxsm\ \GxsmToolbar{Autodisp}.

\Gxsm\ will make use of up to 8192 palette entries automatically.

Here a short description of the PNM file format:

\begin{verbatim}
P3
# CREATOR: The GIMP's PNM Filter Version 1.0
1024 1
255
R
G
B
R
G
B
...
\end{verbatim}


The file is expected to have at least 1024 RGB-entries (R, G, B values in the range 0 \dots 255)
just following the header, but no more than 8192 are accepted.
Use \emph{gimp} to generate this conveniently.
\footnote{1. Palette Editor, 2. fill the picture with 1024x1 pixel,
3. save as \emph{MyPalette.pnm} (ascii-type).}

\Gxsm\ can (re)load a new palette at any time. Choose a new one as
described above and refresh via the \Gxsm\ \GxsmToolbar{AutoDisp}.

Additional custom palette files can be loaded from any location, put
the full path to it into \GxsmPref{Paths}{Paths/UserPalette} and press
OK, then you will find it in the list of available palette files in
\GxsmMenu{View/Palette} on next call of the preferences.

\subsection{Additional Information}
In particular during data acquisition it is convinient to have a 2D representation of the data and a line graph of the last scan line(s). You can get this by activating \GxsmMenu{View/Red profile}. You can also active the display of additional meta-data of the image by \GxsmMenu{View/Side Pan}. That will show you on the different tabs the paramaters/varialbes of the nc-file, Probe Events, User Events, and a list of the Objects.\footnote{The list of objects is currently not autoupdating (Gxsm-2 1.40.0 `Sandy Experiment')}.

% ----------------------------------------

\section{Visualization modes}
\index{Visualization modes}

\label{Gxsm-Visualisation}

The visualization mode can be switched at all times, see
\ref{sec:channels:dialog} for details. 
The default mode is always \GxsmEmph{Grey 2D}.

% ========================================
\subsection{View ``No''}
\index{Channel View!No}

This is not really a visualization mode, it just prevents any display
and saves memory and CPU power if a data channel is just to be
acquired in background in a blind manner.

% ========================================
\subsection{View ``Grey 2D''}
\index{Grey 2D|see{Channel View}}
\index{Channel View!Grey 2D}

The default mode for viewing 2D data. It allows using a simple grey
scale data representation or using of any available false color
palette.  By default the image size is scaled to achieve a good fit on
your screen. But any \GxsmEmph{zoom} (magnifying by number) or
\GxsmEmph{quench} (down-size by 1/number) is possible.

\subsubsection{Window title information}

In the window title of the channel view shows for example:

% \GxsmScreenShot{Grey2DObj}

\GxsmScreenShot{Scan2DVObjects}{Grey 2D view with commonly used
  objects and other gizmos OSD scan info of selected parameters is
  enabled here (red text).  The currently active object (colored
  nodes) is the Rectangle object.  Also enabled at time of this pdf
  data view export is the legend (scale bar and Z legend, below
  option).}

\fbox{\GxsmTT{Ch2: X+ Topo,* Q1/5 Short[2]}}

\begin{description}
\item[\GxsmTT{Ch2}] channel number is 2.
\item[\GxsmTT{X+ Topo,*}] this channel is in data acquiring mode, else here appears the path/filename
\item[\GxsmTT{Q1/5}] the data view is currently down-sized by factor of 5.
\item[\GxsmTT{Short[2]}] the current data is of type short (2 bytes).
\end{description}

\subsubsection{Popup Menu}
\label{Gxsm-TwoD-Popup}
Tthe \GxsmPopup{image}{} by clicking \GxsmMouse{3}{the image}
activates several convenient options and displaying tools:

\begin{description}
\item[Activate] set the current window (channel) to active state (used for math, etc.).
\item[Autodisp] activates this window and performs an automatic scaling.
\item[Mode] switch the channel mode, see \ref{sec:channels:dialog}.
\item[File] load/save scan data in this channel,\\ load/save objects
  (\ref{Gxsm-VObjects}),\\ Get Info (about this scan, it has to be
  (re)loaded before, if it was currently acquired),\\ bring up a print
  dialog for this image,\\ close channel (removes this data from
  memory and closes the window).\\ The current view including all
  objects, legned(s), etc. can be exported as png bitmap and also as
  vector graphics as pdf - -this allows for example to use inkscape to
  edit the meta data and objects or simply remove them.
\item[Edit] Copy, Crop (a marked rectangle, creates a new scan), Zoom
  In/Out (use a rectangle object to select area to zoom in or zoom out
  to, it resizes data into a new scan).
\item[View] change view mode, see \ref{Gxsm-VModes}.
\item[Objects] set type of object to create, see \ref{Gxsm-VObjects}. \GxsmEmph{Remove all} removes all objects.
\item[Events] controls what kind of Events are displayed.
\end{description}


\subsubsection{Objects}
\label{Gxsm-VObjects}
Only in the Grey 2D mode are coordinate measurements and selections
available. Basic objects are Point, Line, Rectangle, PolyLine, Circle,
Ksys (Coordinate System used for atom grid overlays/measuring and also
new basic PAH molecule models can be selected) and Parabel. Some are
available with the prefix ``Show'' in the menu, which means the data
along a path, such as line, circle and in a case of a point the
(possibly) available 3rd layered dimension in depth of scan, is shown
using the profile view (\ref{Gxsm-Profile}).

Select between the different types of objects by the \GxsmPopup{Objects}. You add an object by left-clicking into the 2D image. You can remove an object by clicking with the middle mouse button on it. \footnote{If your notebook or mouse has just two buttons click both at the same time.} You can remove all visible objects by \GxsmPopup{Objects/Remove all}.

By left-clicking on the object handlers you can modify them. In the case that the handlers are two small you can change their size in the menue \GxsmMenu{Settings/Prefereces}. Look for the settings \GxsmPref{GUI}{HandleLineWidth}, \GxsmPref{GUI}{HandleSize} and \GxsmPref{GUI}{ObjectLineWidth}.

For getting the coordinates and Z-value of the mouse position, just
hit the middle mouse button to get a cross hair pointer and move
around to measure. You can switch the shown units from default unit to
pixels using the toggle switch \GxsmPopup{Grey 2D view}{View/Pixels}.

While moving objects, the objects properties coordinates (angle,
length, \dots) are shown in the status line of the window. This is
useful for measurements, etc.. More precise measurements are possible
using the \GxsmEmph{Show Line} object, which is updated on the fly
while moving the line interactively across the image.

Objects are also used to provide some math actions with coordinates.

All object coordinates can be viewed and changed using the object popup
menu (middle mouse button on object). The objects save/load (File menu
of view popup) allows saving objects and reloading those later to any
scan at absolute coordinates (including offset).

While scanning is in progress the current scan-line can be viewed as
profile, use \GxsmPopup{Grey 2D view}{View/red Profile} to
toggle it on and off\footnote{Showing the profile while scanning can
  delay the scanning process on slower machines, so switch it off if
  not needed! To save CPU power you can disable auto-scaling and
  tick marks in the profile view.}.

\subsubsection{Scan Events}
\index{Event}
\index{Scan-Event}
\index{User-Event}
\index{Probe-Event}
\label{Gxsm-Events}
Scan Events, available since GXSM-2.0 V1.6.0, can be any kind of Event
or Experiment related to the current tip position and time. Per
definition a \Gxsm ``Scan Event'' holds a position and possible more
specific information like what happened. I.e a simple bias change
(User-Event-Entry) or a full set of probe data (Prove-Event-Entry).

The number of Events per scan is unlimited. And Events are always
getting attached to the current active scan. The Scan Pop-up menu
Events allows to show Probe Events or User Events. Because there can
be many thousands of Probe Events using the automatic rastered
probing, the displayed markers can be limited in number and distance
from a certain point, indicated by middle-mouse clicking.
 
Probe events are sourced by the SRanger Vector Probing (spectroscopy,
manipulation, etc.), single and automatic (rastered). 
 
Enabling the displaying of Probe Events will add some controls to the
scan window for selecting the range and number of Events to be
displayed. Also the setup for data plotting shows up.

-- to be completed --
 
\subsubsection{Tips}
Define you personal hot-keys with the pop-up, just select an pop-up
entry and press a key!

E.g: Set \GxsmKey{=},\GxsmKey{-} for 
\GxsmPopup{Grey 2D view}{View/zoom in,out}!



% ========================================
\subsection{View ``Surface 3D''}
\index{Surface 3D|see{Channel View}}
\index{Channel View!Surface 3D}
\label{view:surface}

A 3D model of the data, using the Z-value as height, can be viewed
using the GL/Mesa renderer. Hardware acceleration is used by the GL
subsystem if the X-server provides it. This works even while scanning,
but consumes a lot of CPU power and slows the scanning process down.

\GxsmScreenShot{3D-view-GL4-surface}{3D-view example.}


\subsubsection{Pop-up Menu}
The \GxsmPopup{3D view}{} is activated by clicking \GxsmMouse{3}{the
  image}. It includes some of the options already known from the Grey 2D
popup menu, see \ref{Gxsm-TwoD-Popup}.  The popup offers some options for
controlling the 3D view:
\begin{description}
\item[\GxsmTT{GL Options}] Quick access to the display options\\
  \GxsmEmph{Zero Plane} (enable/disable showing of the ``bottom box''),\\
  \GxsmEmph{Mesh} (use meshing instead of solid rendering),\\
\item[\GxsmTT{Scene Setup}] Open the control panel for 3D Scene
  Setup, refer to next section.
\end{description}

Using the mouse for rotation and translation is possible, but a bit
unusual. First, set the window size not too huge and the rendering
resolution (see Scene Setup) above 1 to speed up things. Then press
the left mouse button and move the mouse carefully around for
rotation. Use the middle button for translation.  If you find this
impractical, use the scene setup to set rotation and translation!

\subsubsection{Scene Setup}
\index{3D Scene Setup}
\index{Scene Setup}

\GxsmNote{few preliminary notes on 3D view setup}

This dialog allows a sophisticated 3D rendering configuration.

Mouse/wheel motions: rot, distance, translation. See "Help" buttons!

\begin{description}
	\item[View] Setting of the model (surface) orientation and view.
		The	coordinate system and view controls: scan xyz dimensions are mapped into a 1x1x1 cube (with y may have a aspect if non square) in GL coordinates (GL-XZ is Gxsm-surface plane), Z-scale (absolute, 1=max-value-1x1x1 cube dim or relative angstroem, 1 will scale Z the same as X.)
	
		\begin{description}
			\item[RotationX/YZ] Setting of the model (surface) orientation in world space.
			\item[RotationPreset] Select "Top-View", then elevation is above, and distance "in front of center"... if you rotate object/surface, it changes accordingly. And more default views as named. The "Manual" view will leave XYZ rotation settings untouched vs. pre sets which will reset the rotations to the selected view at any update. 
			\item[LookAt] Predefines look at positions.
			\item[Translation] model translation in world space.
			\item[FoV] Field of View.
			\item[Distance] Distance from model center in "Top-View".
			\item[Elevation] Elevation above model center in "Top-View".
			\item[HeightScaleMode] Height scaling mode. Relative range: 1 equals full data range fits 1x1x1 cube.
			Absolute Range: only meaningful for XYZ data with same units. 1 equals Z scale matching X scale to fit 1x1x1 cube.
			 \GxsmNote{fix typo in GUI} 
			\item[Tskl] N/A
			\item[SliceOffset] Z-Offset of slicing selection in volume mode.
			\item[VertexSource] Flat: no displacement, Direct Height: data is used directly as displacement source,
			ViewMode Height: view mode transformed data is used. You must select the view more before entering the 3D View as data is loaded to the GPU only once. Note: Toggle 3D view back to 2D to update. y: experiments (N/A), Channel-X: Channel X data is used if available. X/Y/Z-slice: slicing views for volume data with value dimension. Switch to Volume Shader (ShadeModel=Volume) in Surface Material Tab. Volume: render 3D data as volume, also use Volume Shader. Scatter: experimental / cube projection, N/A.
			\item[SliceLimiter]: Volume view define the detail and range for view aligned projection planes to be generated. A default cube data set will fit into a 0 .. 1000 range, typically 50..100 slices are sufficient, but you can opt for more detail. Computation intense. Example: $[start = 0, end =1000, step = 10, 0]$ To project beyond you can extend the ranges to for example $[-500, 1500, 10, 0]$ -- may be necessary to cover odd aspect ratio data. Also you can intentionally use it to "cut-open" a volume and only rend par of it - i.e. $[500, 1000, 10, 0]$!
			\item[SlicePlanes] not used at this time.
	\end{description}
	\item[Light] Light sources are ambient and a single "sun" defined by a light direction vector.
	TipView: enable/disable rendering of a virtual "tip".
	
	\item[Surface Material] Color mapping/shading: data is always automatically mapped into a 0..1 full scale range equivalent.
		Gxsm palette is mapped to 0..1 full range. Controls are offset and scale.

		Advanced multi dim data handling (you need a value-dimension with data) if data is in "time" you can run the transformation plugin "multi-dim transpose" to swap time to value.
	
		Slicing and volume view, and the use of transparency settings. Additional transparency for blending. The transfer function for (1-alpha) blending is transparenceoffset+transparencyscale*value. Using Palette color as above.

	\item[Annotations] Title, Labels, etc.
	\item[Render Opt.] 
	\begin{description}
		\item[Ortho] Orthographic projection is used when set to true.
		\item[TessLevel] Use the render option "TessLevel" to control the tesselation level, i.e. the detail of rendering.
		A default triangulated flat base plane spawning a 128x128 grid is used and is subdivided on GPU level up to $64\times$ controlled by the tesselation factor. On GPU level the selected data as used for height field displacements any any resulting given resolution given by the tesselation factor. Only if that happens not to be sufficient you should increase the "BasePlaneGrid" size, but there is usually no need to do so as $128 \times 64$ is bigger than most data sets pixel resolution. Not recommended for performance reasons. You may even want to reduce this for less powerful GL hardware. For this change (BasePlaneGrid only) to take effect you have to click apply and close/reopen the 3D view to rebuild the new geometry.
		\item[BasePlaneGrid] See TessLevel.
		\item[ProbeAtoms] Number "atoms" created for tip visualization.
		\item[Mesh] false: regular 3D rendering. true: use a mesh.
		\item[ShadingMode] Debug Shader only. Debugging and special shading mode/experimental and customizable for GL experts to play with. You can hack even the tess-fragment.glsl shader for custom fun!
		\item[ClearColor] Select the scene backgound color.
		\item[Fog...] Fog settings -- debug shader only, use debug shader, and debug mode "5" to apply fog.
		\item[Cull] Culling means back side triangles of the surface (hidden in top view) are never drawn, mean if you look from below, it's like an roof with no shingles.. faster and normally better. Just disabled in volume/slicing mode obviously..
	
	\end{description}
\end{description}

\GxsmScreenShot{3D-view-GL4-setup-view-surface}{3D Scene View Control, typical settings for a surface terrain like rendering.}
\GxsmScreenShot{3D-view-GL4-setup-surface-material-surface}{3D Scene View Material/Color control, typical settings for a surface terrain like rendering.}
\GxsmScreenShot{3D-view-GL4-setup-render-options-surface}{3D Scene View Options, typical settings for a surface terrain like rendering.}
\GxsmScreenShot{3D-view-GL4-setup-annotations}{Scene annotations.}

\GxsmScreenShot{3D-view-GL4-volume}{Volume rendering example.}


\GxsmScreenShot{3D-view-GL4-setup-view}{3D Scene View Control, typical settings for a volume rendering.}
\GxsmScreenShot{3D-view-GL4-setup-render-options}{3D Scene View Option, typical settings for a volume rendering.}
\GxsmScreenShot{3D-view-GL4-setup-surface-material}{3D Scene View Material/Color control, typical settings for a volume rendering.}



\clearpage

% ========================================
\subsection{The general purpose ``Profile 1D'' view}
\index{1D-profile|see{Channel View}}
\index{Profile 1D|see{Channel View}}
\index{Channel View!Profile 1D}
\label{Gxsm-Profile}
For several purposes Gxsm can show data using the a general purpose
\GxsmEmph{Profile-View} representation (XY-plot).

\GxsmScreenShot{ShowLine}{Profile as used by the Show-Line object (old version gxsm2).}

\GxsmScreenShot{analysis-2d-profile-measuring-1d}{Gxsm3 Profile View of data section with cursors A,B eneabled for measuring.}


In the view-mode \GxsmEmph{Profile 1D} it shows the currently measured
line or the first or via popup menu (\GxsmEmph{Data Select/\dots})
selected line of the data set.
The \GxsmEmph{All} option loads all profiles at once, be careful using
this with huge scan (lots of memory needed). The scaling is (with
respect to shown tickmarks) only correct for all profiles, if the
option \GxsmEmph{Y Scaling/hold} or \GxsmEmph{Y Scaling/expand only}
is activated\footnote{There is a bug with refreshing the toggle
  indicators to the initial setting upon first menu activation, to
  make sure the setting is correct please toggle the option once! Gxsm
  saves the last setting of all these option.}.


\subsubsection{Pop-up Menu}

The popup menu of the \GxsmEmph{Profile View} offer the folowing options:

\begin{description}
\item[File] File operations:
\begin{description}
\item[Open, Save] File handling.
\item[Print] Printing via script. See subsection Profile Printing.
\item[Activate] If used as ``channel view'', to activate this scan.
\end{description}
\item[Options] This menu provides toggle buttons for the following display options:\\
  \begin{description}
  \item[Y lineregression] Enable to use a ``on-the-fly'' line regression.
  \item[Symbols] \GxsmNA
  \item[Legend] \GxsmNA
  \item[Tickmarks] Enable display of tickmarks
  \item[no Gridlines] Disable gridlines
  \end{description}
\item[Y Scaling] Several Y scaling options\footnote{ in case
    no data is appearing, one of the following conditions may be
    present: negative values (using log.) or zero Y or X range.}:
  \begin{description}
  \item[Y logarithmic] Set Y axis to logarithmic scale.
  \item[Y hold] Hold Y scale (prevents automatic rescaling).
  \item[Y expand only] Expand Y scale only.
  \item[Y auto] Force auto scale.
  \item[\dots] A set of manual Y offset shift (move upper/lower bound) 
    and Y zooming (in/out) options.
  \end{description}
  \GxsmHint{For quicker adjusting pull of the menu (drag the menu
  apart by clicking the dashed line) and place it beside!}
\item[X Scaling] \GxsmNA
\item[Cursor] This menu allows to enable up to two cross hair
cursers (blue and green). Use the left mouse button to dragg the
cursors around, click on the vertical cursor line to grab it! The
status line will show (one cursor) the coordinates or (two
cursors) the differential coordinates for measurements.\\
The Cursor menu (tear it off!) allows to auto position the cursor on
local minima or maxima to the left or right of the current position.
Or just to skip to the next/previous data point.
\item[Data Select] Select/walk through a set of multiple data lines or show all.\\
  \GxsmNote{Once selected all, you can't undo this yet, reopen the ``Profile View'' therefore.}
\end{description}



\subsubsection{1D data file format}

Via the \GxsmEmph{File} menu the profie data can be saved in Ascii
format. The file is preceeded by a simple header using commentary
lines and a keyword. The data itself follows using the following format:

Height Profile Data:

\begin{verbatim}
# ASC LineProfile Data
# Date: Tue Jun 18 17:21:42 2002
# FileName: /tmp/HutProfile.asc
# Len   = 405.843A
# tStart= 1016686983s
# tEnd  = 1016687662s
# phi   = 39.9441\B0
# dX    = 0.235244A
# dY    = -0.197002A
# dZ    = 0.0139465A
# Xo    = -1079.55A
# Yo    = 0A
# Anz   = 1323
# NSets = 1
# Xunit = A
# Xalias= AA
# Xlabel= L
# Yunit = A
# Yalias= AA
# Ylabel= Z
# Title = title not set
# Format= X[index*LineLen/Nx] Z[DA]... InUnit: X[Xunit] Z[Zunit]...
0 -103 InUnits: 0 -1.43649
0.30676 -105 InUnits: 0.306992 -1.46439
0.613519 -107 InUnits: 0.613983 -1.49228
0.920279 -107 InUnits: 0.920975 -1.49228
1.22704 -106 InUnits: 1.22797 -1.47833
1.5338 -104 InUnits: 1.53496 -1.45044
...
\end{verbatim}

STS Data:

\begin{verbatim}
# ASC LineProfile Data
# Date: Tue Jun 18 17:24:59 2002
# FileName: /tmp/STS.asc
# Len   = 0A
# tStart= 1024442686s
# tEnd  = 1024442686s
# phi   = 0\B0
# dX    = 0A
# dY    = 0A
# dZ    = 1A
# Xo    = 0A
# Yo    = 0A
# Anz   = 500
# NSets = 2
# Xunit = V
# Xalias= V
# Xlabel= Bias
# Yunit = nA
# Yalias= nA
# Ylabel= I
# Title = STS [local Spectroscopy]
# Format= X[index*LineLen/Nx] Z[DA]... InUnit: X[Xunit] Z[Zunit]...
0 0 0 InUnits: -2 0 0
0.02 0 0 InUnits: -1.99198 0 0
0.04 0 0 InUnits: -1.98397 0 0
0.06 0 0 InUnits: -1.97595 0 0
0.08 0 0 InUnits: -1.96794 0 0
...
\end{verbatim}


Simple Shell Script for extracting just X and Y data:

\begin{verbatim}
#!/bin/bash

for file do
 awk '{ if ($1 != "#") { print $5, $6}}' $file > $file.dat1
done
\end{verbatim}

Gxsm can reload this type of files using just the \GxsmEmph{Open}
command. The file should have the \filename{.asc} extension to be
recognized as a profile data set.


\subsubsection{Profile Printing}
Starting from the \GxsmEmph{File} menu you will find the
\GxsmEmph{Print}-entry. From here you have access to six user commands.
This commands are simple shell scripts that receive a temporarily created
version of the 1D profile as argument and can do various things with this 
profile (not only print). There are six example scripts in the profileplot
subdirectory. 
\begin{enumerate}
\item The first script runs \GxsmEmph{gri} to process your profile and will
create a postscript file of your print.
\item The second script runs \GxsmEmph{gri} to process your profile and will
create a postscript file of your print. The data are modified with an
embedded Python script to calculate STS.
\item The third script runs \GxsmEmph{gri} to process your profile and will
create a postscript file of your print. The y-axis is log'ed here.
\item The fourth script runs \GxsmEmph{gri} to process your profile and will
create a postscript file of your print. This is a copy of the first script.
Use this to experiment.
\item The fifth script runs \GxsmEmph{xmgrace} to show your profile.
Remember that Gxsm will freeze while xmgrace is running, so don't use this
option as long as a scan is running.
\item The sixth script runs \GxsmEmph{python-matplot} to show your profile.
You can export to many bitmap formats from this. Python-matplot is not (yet;
check it out) part of debian testing (Feb 05)\footnote{Follow the instructions
on the matplotlib-website for installing. It's easy. Simply add one line to /etc/apt/sources.list
and run apt-get update \&\& apt-get install python-matplotlib}.
\end{enumerate}

These scripts can run standalone (i.\ e. without Gxsm running). Simply run them with an
ASCII-1D-profile as argument like this:

/path/gxsmplot1d-1.gri.sh myprofile.asc

One can think of many different things these scripts are capable of. They
can run any calculation on the profiles. These scripts are designed to run
together with the \GxsmEmph{Show Line} option, not with the 1-D profile view
in the channel selector. If you use the latter you will always receive a
plot of line 0.


You can run your own scripts by modifying the entries
User/command[1..6] in the preferences.



%%% Local Variables: 
%%% mode: latex
%%% TeX-master: "Gxsm-main"
%%% End: 
